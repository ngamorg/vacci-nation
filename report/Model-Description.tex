\documentclass[11pt]{article}
\usepackage{geometry}                
\geometry{a4paper}                   

\usepackage[latin1]{inputenc}

\usepackage[colorlinks, linkcolor=blue, citecolor=blue, filecolor=blue, urlcolor=blue]{hyperref}
\usepackage{hyperref}
\usepackage[nottoc,notlot,notlof]{tocbibind}

\usepackage{graphicx}
\usepackage{amssymb}
\usepackage{epstopdf}
\usepackage[comma]{natbib}
\usepackage{amssymb, amsmath}
\renewcommand{\vec}[1]{\mathbf{#1}}

\DeclareGraphicsRule{.tif}{png}{.png}{`convert #1 `dirname #1`/`basename #1 .tif`.png}

\title{Epidemiology \& Immunization}
\author{Julius Siebenaller, Jannes Hühnerbein, Benjamin Gundersen, Nicolas Antunes Morgado}
\date{November 2019} 

\begin{document}



\section{Description of the Model}

% Considerations regarding the disease and the modelled time-horizon:
% Assume an outbreak of a severe disease with the possibility to receive a vaccination that works immediately when not yet infected. 

The general assumptions of our model are as follows:
\begin{itemize}
\item{The infectious disease is modelled as a time-homogeneous Markov chain with a finite state space according to an SIVR-process with 'vaccinated', respectively 'recovered' representing the final or absorbing states with permanent immunity. The population is assumed to stay constant.}
\item{Agents are assumed to be rational utility-maximisers endowed with complete information about their immediate neighborhood. They decide in each time-step whether or not to update their vaccination decision and hence if they will get vaccinated employing a mixed strategy.}
\item{When facing the vaccination decision, each agent samples the health status of its direct neighborhood and deduces the probability of infection during the next epoch from the amount of infected neighbors.}
\item{The time-horizon of an agent is bounded, in the sense that only a certain amount of time-steps is considered in the calculation of expected utility to be maximized through the vaccination-decision.}
\item{Agents have a group affiliation to either be skeptical (group: 'skepticals') or trust (group: 'trusters') that vaccines are safe. The group membership of an agent determines the subjective costs they assign to vaccination, while the costs of infection are assumed to be equal between groups.}
\item{The population structure is modelled as a scale-free network, relying on a Barabási-Albert model. We assume the two groups to represent scale-free networks on their own. These two networks are then combined assuming that closeness in a group is larger than between groups.}
\end{itemize}
The work of SHIM ET AL motivates the agent decision process implemented in the model and is partly derived from their analysis. Still, the approach developed in the present work deviates from it in several respects. First off, a discrete-time rather than continuous time Markov process is used to describe disease dynamics. This choice is justified both on simplicity grounds and because at least part of the dynamics -such as meeting other agents- are inherently discrete events. Secondly, individual transition probabilities are based on the local instantaneous incidence of infection rather than the steady endemic distribution. The latter is justified by the fact that the present model employs a network structure for the population rather than a lumped, well mixed approximation. Moreover, the derivation of the vaccination policy does not aim to obtain the Nash equilibrium strategy for the population but rather a locally optimal update rule under non-equilibrium conditions.
%Restriction to the assumption of complete information, but can on the other hand justify the Markov process rule

Concerning the epidemiological part of our model, the spread of the disease in the population follows an SIVR-process. Thus, the population is divided into (1) susceptible (2) infectious (3) recovered and (4) vaccinated subpopulations. The subpopulations are the same for the two belief-groups of scepticals and trusters. Susceptible agents can transition to the infectious state, \(S \rightarrow{} I \), when the infection from one of their neighbours spreads according to the parameter \(\beta\). If deciding for a vaccination, the agent transitions according to the rule \(S \rightarrow{} V \). An agent arrives at the decision to vaccinate with probability \(p_{vacc}(T)\) in every time-step \(T\). Infectious agents recover by the probability \(\gamma_{k}\), \(I \rightarrow{} R\). Recovered, respectively vaccinated, agents stay in their respective state which can be regarded as isomorphic with the exception of how these states are reached. Both recovered and vaccinated agents are assumed to become fully immunised. A constant population is assumed in a way that no entries, via birth, or exits, via death, are taken into account.\footnote{A more detailed version of the SIVR-model can be found in XYZ or ABC.} The evolution of the system is summarised in the stochastic matrix \(\mathbf{Q_{k}}\) and the health-state vector \(\mathbf{x_{k}(t)}\). The disease-specific parameters and transmission rates are outlined below:
\begin{equation}\label{eq:1}
	 \mathbf{x_{k}(t+1)} = \(\mathbf{Q_{k}}\) *  \mathbf{x_{k}(t)},
\end{equation}
and 
\begin{equation}\label{eq:2}
	 \(\mathbf{Q_{k}}\) = 
	 \left (\begin{array}{cccc} p_{k}(S\rightarrow{}S)&p_{k}(I\rightarrow{}S)&p_{k}(R\rightarrow{}S)&p_{k}(V\rightarrow{}S) \\
	 	p_{k}(S\rightarrow{}I)&p_{k}(I\rightarrow{}I)&p_{k}(R\rightarrow{}I)&p_{k}(V\rightarrow{}I) \\
		p_{k}(S\rightarrow{}R)&p_{k}(I\rightarrow{}R)&p_{k}(R\rightarrow{}R)&p_{k}(V\rightarrow{}R) \\
		p_{k}(S\rightarrow{}V)&p_{k}(I\rightarrow{}V)&p_{k}(R\rightarrow{}V)&p_{k}(V\rightarrow{}V) \\
	  \end{array} \right)
\end{equation}
In particular: 
\begin{equation}\label{eq:3}			% time-dependency of Q_{k}
	 \(\mathbf{Q_{k}}\) = 
	 \left (\begin{array}{cccc} 1-\lambda_{k}&0&0&0 \\
	 	\lambda_{k}&1-\gamma_{k}&0&0 \\
		0&\gamma_{k}&1&0 \\
		0&0&0&1 \\
	  \end{array} \right)
\end{equation}
with 
\begin{equation}\label{eq:7}
	\mathbf{x_{k}(t=0)}=\left(\begin{array}{cccc} 1-p_{vacc},&0,&0,&p_{vacc}  \end{array} \right),
\end{equation} 
\begin{equation}\label{eq:7}
	\mathbf{x_{k}(t=0)}=\left(\begin{array}{cccc} 1-\beta^{initial}-p_{vacc}^{initial} ,&\beta^{initial},&0,&p_{vacc}^{initial}  \end{array} \right)
\end{equation} 
representing the state at \(t=0\), relying on the initial probabilities of infection and vaccination. 

% INITIAL STATES



As outlined above, \(\beta\) is an individual's probability of getting infected by one infected agent in the neighbourhood. \(\beta\) is assumed to be disease-specific and thus of constant value, independent of the agent. The actual rate of infection \(\lambda_{k}\) results from an agent sampling the neighbourhood and deriving the probability of infection given the edges that are connected with an infected note: 
\begin{equation}\label{eq:5}
	\lambda_{k}=1-(1-\beta)^{n_{k}},
\end{equation}
with \(n_{k}\) being the amount of infected neighbours of agent \(k\). %, which takes the following form: xyz % age-specific 

The agent, being a rational utility-maximiser, faces the vaccination decision by solving the optimisation problem:
\begin{equation}\label{eq:10}
	\max p_{vacc}\{U(x)\}
\end{equation}
Here, the utility function takes the subsequent form, following the Bellman equation for a discrete-time Markov process: % reference for Bellmann equation
\begin{equation}\label{eq:6}
	\begin{split}
		U_{k}(x) = \sum_{t=0}^{T}\mathbf{x_{k}(t)} * \frac{\mathbf{f_k(t)}}{(1+r)^t}
	\end{split}
\end{equation}
Where \(\mathbf{f_{k}(t)}\) is the payoff vector of agent \(k\) with 		% Payoff-vectors
\begin{equation}\label{eq:8}
	\mathbf{f_{k}(t=0)}=\left( \begin{array}{cccc}0&0&0&-Cv_{k} \end{array} \right)^T,	
	% other suggestions for transpose T
\end{equation} 
and
\begin{equation}\label{eq:9}
\(\mathbf{f_{k}(t=0)}=\left( \begin{array}{cccc} 0&-Ci_{k}&0&0 \end{array} \right)^T  
\end{equation} represent the immediate vaccination payoffs, respectively the payoffs per time as result from infection.  
The costs of vaccination, among others costs for the drug or possible side-effects, are included in \(Cv_{k}\). Besides, \(Ci_{k}\) captures the costs of infection, including without limitation expenditures for curative health-care absence from work or reduced personal well-being. Agents discount the payoffs of future events with a discount rate \(r\) and an agent's time horizon is shown via \(T\). \\
Equation \eqref{eq:XYZ} can be rewritten using equation \eqref{eq:EFG}:
\begin{equation}\label{eq:11}
	U_{k}(x)=\mathbf{x_{k}(t=0)} * \mathbf{f_{k}(t=0)} + \sum_{t=1}^{T}\mathbf{Q_{k}}^{t} * \frac{\mathbf{f_k(t)}}{(1+r)^t} * \mathbf{x_{k}(t=0)}
\end{equation}
% U_k(x) = ( f_k(t = 0) + sum_{t = 1}^{t = T} Q^t * f_k(t) / (1 + r)^t ) * x_k(0)                      (3)
In order to compute the power  \(\mathbf(Q_{k})^t\)%Q^t
, the matrix can be diagonalized: 
\begin{equation}\label{eq:12}
	\mathbf{Q_{k}}=\mathbf{P_{k}}^{-1}\mathbf{L_{k}P_{k}}
%Q_k = P_k^-1 * L_k * P_k
\end{equation}
And we can rewrite \eqref{eq:10} as follows:
\begin{equation}\label{eq:13}
	U_{k}=\mathbf{x_{k}(0)}\mathbf{f_{k}(0)}+\mathbf{x_{k}(0)}*\mathbf{P_{k}^{-1}}*\sum_{t=1}^{T}\frac{\mathbf{L_{k}}}{(1+r)^t}*\mathbf{P_{k}}*\mathbf{f_{k}(t)}
% U_k(x) = ( f_k(t = 0) + P_k^-1 * sum_{t = 1}^{t = T} (L_k / (1 + r))^t * P_k * f_k(t) ) * x_k(0)     (4)
\end{equation}
Finally, performing the required computations we obtain: 
\begin{equation}\label{eq:14}
	U_{k}(x)=-p_{vacc} * Cv_{k} - (1-p_{vacc}) * Cnotv_{k}
% U_k(x) = -pvacc * Cv_k - (1 - pvacc) * Cnotv_k                                                       (5)
\end{equation}
Where: 
\begin{equation}\label{eq:15}
	Cnotv_{k}=Ci_{k}*\frac{\lambda_{k}}{\gamma_{k}-\lambda_{k}}*\bigg(\sum_{t = 1}^{t = T} \Big(\frac{1-\lambda_k}{1+r}\Big)^t - \sum_{t = 1}^{t = T}\Big(\frac{1-\gamma_k}{1+r}\Big)^t\bigg)
% U_k(x) = -pvacc * Cv_k - (1 - pvacc) * Cnotv_k                                                       (5)
\end{equation} 
Equation \eqref{eq:13} is linear in \(p_{vacc}\), so it has only one global maximum which depends on the relative cost of each each decision. In particular, the preferred strategies are the following:
\begin{equation}
	\left\{\begin{array}{@{}lr@{}}
        \multirow{p_{vacc}=0,} & \text{for }Cv_{k}>Cnotv_k}\\
        {p_{vacc}=1,} & \text{for }Cnotv_{k}>Cv_k}\\
        {p_{vacc}\in(0,1),} & \text{for }Cv_{k}=Cnotv_k}\\
        \end{array}\right\}
\end{equation}
Ref.:
\cite{shim2012}



INITIAL SETUP
IN OUR CALCUALTIONS, WE USE THE PARAMETERS AS OUTLINED BELOW:

NETWORK
We decided on utilising 
The hierarchical configuration model in STEGEHUIS ET AL (2016) serves as a basis for the model which is employed here. The two belief-groups are assumed to be closely connected within a group and are identified as sub-communities of the whole population. Relying on a small world model, each community is built by formulating a Watts-Strogatz network (WATTS, STROGATZ). % FEATURES

NEIGHBOURS OF NEIGHBOURS: SHI ET AL (2017)


\section{Computational model}
All our calculations are carried out in Python. Especially numpy, matplotlib for plotting, networkx for the creation of our network. 


\end{document}  



 
